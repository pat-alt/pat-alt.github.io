$-- Contains the document class declaration and options.
$-- By default we provide the identical document class that Pandoc provides, implementing many features.
$-- If provide this partial in your format, you will need to either implement support for the usaul document class options in addition to other one
$-- or be aware that pandoc supported options (e.g. font-size, paper-size, classoption, etc) will not be supported in your format.
$--

\documentclass[
$if(fontsize)$
  $fontsize$,
$endif$
$if(lang)$
  $babel-lang$,
$endif$
$if(papersize)$
  $papersize$paper,
$endif$
$for(classoption)$
  $classoption$$sep$,$endfor$
]
{$documentclass$}

$if(CJKmainfont)$
\PassOptionsToPackage{space}{xeCJK}
$endif$

% default font
\usepackage{ebgaramond-maths}

$if(linestretch)$
\usepackage{setspace}
$endif$
\usepackage{iftex}
\ifPDFTeX
  \usepackage[$if(fontenc)$$fontenc$$else$T1$endif$]{fontenc}
  \usepackage[utf8]{inputenc}
  \usepackage{textcomp} % provide euro and other symbols
\else % if luatex or xetex
$if(mathspec)$
  \ifXeTeX
    \usepackage{mathspec} % this also loads fontspec
  \else
    \usepackage{unicode-math} % this also loads fontspec
  \fi
$else$
  \usepackage{unicode-math}
$endif$
  \defaultfontfeatures{Scale=MatchLowercase}$-- must come before Beamer theme
  \defaultfontfeatures[\rmfamily]{Ligatures=TeX,Scale=1}
\fi

$if(fontfamily)$
\usepackage[$for(fontfamilyoptions)$$fontfamilyoptions$$sep$,$endfor$]{$fontfamily$}
%\usepackage[T1]{fontenc}
$endif$
% xetex/luatex font selection
$if(mainfont)$
  \setmainfont[$for(mainfontoptions)$$mainfontoptions$$sep$,$endfor$]{$mainfont$}
$endif$
$if(sansfont)$
  \setsansfont[$for(sansfontoptions)$$sansfontoptions$$sep$,$endfor$]{$sansfont$}
$endif$
$if(monofont)$
  \setmonofont[$for(monofontoptions)$$monofontoptions$$sep$,$endfor$]{$monofont$}
$endif$



$if(csl-refs)$
% definitions for citeproc citations
\NewDocumentCommand\citeproctext{}{}
\NewDocumentCommand\citeproc{mm}{%
  \begingroup\def\citeproctext{#2}\cite{#1}\endgroup}
\makeatletter
 % allow citations to break across lines
 \let\@cite@ofmt\@firstofone
 % avoid brackets around text for \cite:
 \def\@biblabel#1{}
 \def\@cite#1#2{{#1\if@tempswa , #2\fi}}
\makeatother
\newlength{\cslhangindent}
\setlength{\cslhangindent}{1.5em}
\newlength{\csllabelwidth}
\setlength{\csllabelwidth}{3em}
\newenvironment{CSLReferences}[2] % #1 hanging-indent, #2 entry-spacing
 {\begin{list}{}{%
  \setlength{\itemindent}{0pt}
  \setlength{\leftmargin}{0pt}
  \setlength{\parsep}{0pt}
  % turn on hanging indent if param 1 is 1
  \ifodd #1
   \setlength{\leftmargin}{\cslhangindent}
   \setlength{\itemindent}{-1\cslhangindent}
  \fi
  % set entry spacing
  \setlength{\itemsep}{#2\baselineskip}}}
 {\end{list}}
\usepackage{calc}
\newcommand{\CSLBlock}[1]{\hfill\break#1\hfill\break}
\newcommand{\CSLLeftMargin}[1]{\parbox[t]{\csllabelwidth}{\strut#1\strut}}
\newcommand{\CSLRightInline}[1]{\parbox[t]{\linewidth - \csllabelwidth}{\strut#1\strut}}
\newcommand{\CSLIndent}[1]{\hspace{\cslhangindent}#1}
$endif$

$if(euro)$
  \newcommand{\euro}{€}
$endif$
$if(mainfont)$
    \setmainfont[$for(mainfontoptions)$$mainfontoptions$$sep$,$endfor$]{$mainfont$}
$endif$
$if(sansfont)$
    \setsansfont[$for(sansfontoptions)$$sansfontoptions$$sep$,$endfor$]{$sansfont$}
$endif$
$if(monofont)$
    \setmonofont[Mapping=tex-ansi$if(monofontoptions)$,$for(monofontoptions)$$monofontoptions$$sep$,$endfor$$endif$]{$monofont$}
$endif$
$if(mathfont)$
    \setmathfont(Digits,Latin,Greek)[$for(mathfontoptions)$$mathfontoptions$$sep$,$endfor$]{$mathfont$}
$endif$
$if(CJKmainfont)$
  \ifXeTeX
    \usepackage{xeCJK}
    \setCJKmainfont[$for(CJKoptions)$$CJKoptions$$sep$,$endfor$]{$CJKmainfont$}
    $if(CJKsansfont)$
      \setCJKsansfont[$for(CJKoptions)$$CJKoptions$$sep$,$endfor$]{$CJKsansfont$}
    $endif$
    $if(CJKmonofont)$
      \setCJKmonofont[$for(CJKoptions)$$CJKoptions$$sep$,$endfor$]{$CJKmonofont$}
    $endif$
  \fi
$endif$
$if(luatexjapresetoptions)$
  \ifLuaTeX
    \usepackage[$for(luatexjapresetoptions)$$luatexjapresetoptions$$sep$,$endfor$]{luatexja-preset}
  \fi
$endif$
$if(CJKmainfont)$
  \ifLuaTeX
    \usepackage[$for(luatexjafontspecoptions)$$luatexjafontspecoptions$$sep$,$endfor$]{luatexja-fontspec}
    \setmainjfont[$for(CJKoptions)$$CJKoptions$$sep$,$endfor$]{$CJKmainfont$}
  \fi
$endif$
% use upquote if available, for straight quotes in verbatim environments
\IfFileExists{upquote.sty}{\usepackage{upquote}}{}
% use microtype if available
\IfFileExists{microtype.sty}{%
\usepackage{microtype}
\UseMicrotypeSet[protrusion]{basicmath} % disable protrusion for tt fonts
}{}
$if(geometry)$
\usepackage[$for(geometry)$$geometry$$sep$,$endfor$]{geometry}
$endif$


$if(lang)$
\ifnum 0\ifxetex 1\fi\ifluatex 1\fi=0 % if pdftex
  \usepackage[shorthands=off,$for(babel-otherlangs)$$babel-otherlangs$,$endfor$main=$babel-lang$]{babel}
$if(babel-newcommands)$
  $babel-newcommands$
$endif$
\else
  \usepackage{polyglossia}
  \setmainlanguage[$polyglossia-lang.options$]{$polyglossia-lang.name$}
$for(polyglossia-otherlangs)$
  \setotherlanguage[$polyglossia-otherlangs.options$]{$polyglossia-otherlangs.name$}
$endfor$
\fi
$endif$
